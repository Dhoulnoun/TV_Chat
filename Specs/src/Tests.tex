%--------------------------------------
% Spécifications du chat Felix - Camix
%
% CU
%--------------------------------------

\section{Tests}
\label{sec:t}

\medskip
Dans le cadre du projet, nous avons été amenés à réaliser des tests unitaires et des tests d'intégration.
Ces tests doivent prendre en compte les demandes du client.
Ainsi que toute mise à jour apportée par ce dernier.

\subsection{Specifications demandée par le client}
\label{sec:t:specclient}

Vous pourrez retrouver les spécifications demandées par le client dans la section \ref{sec:cu}.
En revanche dans le cadre du projet, nous avons rédigé des spécifications liées à l'entrée dans le chat en fonction de la mise à jour du client.


\subsection{Entrer dans le chat - MIS A JOUR}
\label{sec:t:entrerchat}

\noindent
Cas d'utilisation : Entrer dans le chat.\\
Résumé : Un utilisateur entre dans le chat en lançant le logiciel client du chat.\\
Acteur principal : Utilisateur.\\
Acteurs secondaires : Les autres utilisateurs du chat.\\
Pré-conditions : Un logiciel serveur (Camix) est lancé et accessible par le réseau TCP/IP.

\medskip
\textbf{Spécifications fonctionnelles avec variantes} :
\begin{enumerate}
\item L'utilisateur lance l'exécution du composant Felix.
\item Felix affiche la vue de connexion au chat.
\item L'utilisateur demande à se connecter.
\item Felix affiche un message de connexion.
\item Retour sur le point d'origine. (2)
    \item Camix inscrit l'utilisateur dans le canal par défaut. (place publique)
    \item Camix envoie un message d'arrivée à tous les utilisateurs du canal par défaut.
    \item Chaque composant Felix affiche le message d'arrivée.
    \item Camix transmet au composant Felix de l'utilisateur un message d'accueil dans le chat.
    \item Felix ferme la vue de connexion au chat.
    \item Felix affiche la vue du chat.
    \item Felix affiche le message d'accueil dans le chat.
\end{enumerate}

\textbf{Variantes :[Modifications de l'addresse IP du serveur]}
\begin{enumerate}
    \item[3.a.1.] L'utilisateur modifie l'adresse IP du serveur.
    \item[3.a.2.] Retour à l'étape 3. de la spécification fonctionnelle.
\end{enumerate}

\textbf{Variantes :[Modification du port du serveur]}
\begin{enumerate}
    \item[3.b.1.] L'utilisateur modifie le port du serveur.
    \item[3.b.2.] Retour à l'étape 3. de la spécification fonctionnelle.
\end{enumerate}

\textbf{Variantes :[Connexion impossible]}
\begin{enumerate}
    \item[6.a.1.] Felix affiche un message de connexion impossible.
    \item[6.a.2.] Retour à l'étape 3. de la spécification fonctionnelle.
\end{enumerate}



